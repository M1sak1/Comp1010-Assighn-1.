\documentclass[12pt, times new roman]{article}
\title{\textbf{First document}}
\author{Josh Corrigan}
\begin{document}
\maketitle
\begin{abstract}
This is a small simple example of what will be in the SCIENTIFIC DOCUMENT!!!!!!
\end{abstract}
\section{Introduction}
	Does the introduction extend the title, given more background that can be put the work into the context? \newline
	Does the intrduction contain a general overview of the report and clearly explains the overall aim of the work and a comment about whether that aim was achieved?
\section{Body}
	How does digital media development affect the growing changes in specific hardware developement. \newline
	Refrences style use bib style, most scientific documents will have a bib style refrence already in a bib document.
\newline
\newline
\newline
\newline
\section{josh's question}
The development of hardware of a specific purpose is a filed heavily influenced by the always advancing nature of digital media development. The development of digital media has presented itself as a global communication tool able to be used in a multitude of different ways for varying purposes. An example of this is the online tutorials that convey knowledge of any type allowing anyone with access to the internet to learn information and follow instructions from a device with the services such as Skill share or Youtube that offer tutorials and useful information. Social media has heavily influenced the way we consume media and interact with eachother and due to it being such a large market many specialised systems are being designed for the purpose of social media. \newline
\subsection{Online tutorials}
A Online Tutorial consist of a person recording and explaining the actions taken to achieve a result, where that footage will be edited, polished and uploaded online where watchers of the tutorial or users can watch those recordings and sometimes follow the tutorial mimicing the actions and trying to create the same thing as being shown. This can be achieved through the use of hardware devices notiably a camera or screen capture card which will record the tutorial allowing users to visually see what to do, additionally the use of a microphone provides the user with auditory instructions and more information than what is being show as the author of the tutorial will most likely be talking about what to do and why we do it a certain way presenting a deeper understanding of the material. The user can view the tutorials on a multitude of devices and it depends on the type of tutorial to determine what device would be the best option as something handheld and small would be much better for a cooking tutorial than a desktop while a desktop would be advantages for a computer program tutorial or something similar that requires the access of a computer to implement.  \newline 
A tutorial that requires both hands to mimic and would be inefficient when having the tutorial on a surface not immediently close to the user presents concearns of walking back and forth having the pause and unpause the video one action at a time seems very time consuming and ineffective. Google Glass is a device marketed as a hands free way to view information and has speaker capabilities that would present itself as a useful tool for users mimicing tutorials as it would allow them to follow along concurrently and have both hands free. Studies done in the useful ness of Google Glass's when used for online tutorials has concluded Google Glass intuitively has some advantages. The display is wearable, leaving both hands free to interact with the object of interest. However, in our study the participants overwhelmingly preferred to use a tablet when using a tutorial to recreate a model [Tools for online tutorials,2018]. The Google Glass if made more user friendly and the users were provided with training on how to use the device, specialised hardware like the Google Glass will become incredibly useful with the posibility of completely replacing the microphone and camera a author uses to create the tutorial. The more time spend optimising how we learn and share information will naturally form a society that knows more skills therefore the studies relating to the connection between digital media development and specific hardware development justify the time spend developing these new technologies.  \newline


	*The lifecycle of an online tutorial involves an author capturing the  
steps necessary to complete a task, and then assembling, editing, and annotating a
curated set of captured material into a multimedia representation.    //tools for online tutorial
\newline
\subsection{Social Media}
*Social media
	Daily experiences are increasingly experienced through digital devices and smartphones in particular have become
a predominant site for consuming, sharing and searching for media contents. //Guess whats on my screen
	 Today mobile phones are the most ubiquitous personal computing devices in the planet covering around 96.8 percent  of the world population (ITU World Telecommunication/ICT Indicators Database 2013)  //Social BOTS

		*facebook
	*clickbait 


	due to its pervasiveness and ability to passively log
	users’ context, smartphones offer the opportunity of collecting data at an unprecedented scale. //Sensing and modeling Human behaviour 

		*Phones and the multitude of devices that allow you to access social media 
\begin{itemize}
	\item What specific hardware is being developed for the purpose of digital media development 
	\item What considerations were minded while developing these hardware and what specific negative reactions were found 
	\item secondary purposes of digital media and how that affects development.
	\item Justify why the relationship between digital media development and specific hardware is important in relation to your question
	\item description of relevant work in the field published in journals and confrences
	\item brief review of the key literature??? (course books or refrences?)  
	\item relevant work included as examples 
	\item link back to intro statement  
	\item refrences 
\end{itemize}
\section{Conclusion}
\section{Glossary}
\begin{itemize}
	\item arxiv.org/abs/1801.08997 
@article{DBLP:journals/corr/abs-1801-08997,
author    = {Scott A. Carter and
               Pernilla Qvarfordt and
               Matthew Cooper and
               Aki Komori and
               Ville M{\"{a}}kel{\"{a}}},
title     = {Tools for online tutorials: comparing capture devices, tutorial representations,
               and access devices},
journal   = {CoRR},
volume    = {abs/1801.08997},
year      = {2018}
}
	\item arxiv.org/abs/1707.07592 (The spread of low-credibility content by social bots) 								  $$$$
	\item arxiv.org/abs/1702.01181 (Sensing and Modeling Human Behaviour Using Social Media and Mobile Data) 				  $$$$
	\item arxiv.org/abs/1901.02701 (Guess whats on my screen? Clustering Smartphone Screenshots with Active Learning) 			  $$$$
	\item ITU World Telecommunication/ICT Indicators Database (2013).
\end{itemize}
%this is a comment defo use it for a article 
% businesses try to enhance the experience, notable examples are facebooks share function and SnapChats message deletion (for social media expanded)
\end{document}
 