\documentclass[12pt, times new roman]{article}
\title{\textbf{First document}}
\author{Josh Corrigan}
\begin{document}
\maketitle
\begin{abstract}
This is a small simple example of what will be in the SCIENTIFIC DOCUMENT!!!!!!
\end{abstract}
\section{Introduction}
	Does the introduction extend the title, given more background that can be put the work into the context? \newline
	Does the intrduction contain a general overview of the report and clearly explains the overall aim of the work and a comment about whether that aim was achieved?
\section{Body}
	How does digital media development affect the growing changes in specific hardware developement. \newline
	Refrences style use bib style, most scientific documents will have a bib style refrence already in a bib document.
\newline
\newline
\newline
\newline
\section{josh's question}
The development of hardware of a specific purpose is a filed heavily influenced by the always advancing nature of digital media development. The development of digital media has presented itself as a global communication tool able to be used in a multitude of different ways for varying purposes. An example of this is the online tutorials that convey knowledge of any type allowing anyone with access to the internet to learn information and follow instructions from a device with the services such as Skill share or Youtube that offer tutorials and useful information. Social media has heavily influenced the way we consume media and interact with eachother and due to it being such a large market many specialised systems are being designed for the purpose of social media. \newline
\subsection{Online tutorials}
A Online Tutorial consist of a person recording and explaining the actions taken to achieve a result, where that footage will be edited, polished and uploaded online where watchers of the tutorial or users can watch those recordings and sometimes follow the tutorial mimicing the actions and trying to create the same thing as being shown. This can be achieved through the use of hardware devices notiably a camera or screen capture card which will record the tutorial allowing users to visually see what to do, additionally the use of a microphone provides the user with auditory instructions and more information than what is being show as the author of the tutorial will most likely be talking about what to do and why we do it a certain way presenting a deeper understanding of the material. The user can view the tutorials on a multitude of devices and it depends on the type of tutorial to determine what device would be the best option as something handheld and small would be much better for a cooking tutorial than a desktop while a desktop would be advantages for a computer program tutorial or something similar that requires the access of a computer to implement.  \newline 
A tutorial that requires both hands to mimic and would be inefficient when having the tutorial on a surface not immediently close to the user presents concearns of walking back and forth having the pause and unpause the video one action at a time seems very time consuming and ineffective. Google Glass is a device marketed as a hands free way to view information and has speaker capabilities that would present itself as a useful tool for users mimicing tutorials as it would allow them to follow along concurrently and have both hands free. Studies done in the useful ness of Google Glass's when used for online tutorials has concluded Google Glass intuitively has some advantages. The display is wearable, leaving both hands free to interact with the object of interest. However, in our study the participants overwhelmingly preferred to use a tablet when using a tutorial to recreate a model [Tools for online tutorials,2018]. The Google Glass if made more user friendly and the users were provided with training on how to use the device, specialised hardware like the Google Glass will become incredibly useful with the posibility of completely replacing the microphone and camera a author uses to create the tutorial. The more time spend optimising how we learn and share information will naturally form a society that knows more skills therefore the studies relating to the connection between digital media development and specific hardware development justify the time spend developing these new technologies.  \newline
\newline
\subsection{Social Media}
Social media is any platform that allows the sharing and recieving of content from others online which has gotten so popular it has replaced many forms of in-person communication, Daily experiences are increasingly experienced through digital devices and smartphones in particular have become a predominant site for consuming, sharing and searching for media contents [Guess whats on ny screen]. Smartphone brand Samsung has installed Facebook as a permanent  app its phone's which illustrate its popularity as a social media and justifys the cost of specialised systems developed for facebook. Deepface is a system designed by and for Facebook that uses the pictures provided by Facebook users to approximate who is what person in a picture uploaded to Facebook and provide the persons name. This present privacy concernes as Facebook has designed a system that can pick out people in a croud with a 97.35 percent accuracy rate, what this means for people who dont want their facebook name shared with anyone outside who they want becomes irrelevant if the persons face is posted to a Facebook page that is viewed by a lot of users any one of those users can know their facebook name and harras them through multiple accounts.\newline 
Mobile phones are the most ubiquitous personal computing devices in the planet covering around 96.8 percent of the world population[ITU World Telecommunication 2013], this data is from 2013 and assuming that percentage will only rise means most of the population are using mobile or smart phones where the user is always passivly giving away their likes, interests, hobbies, location, etc and due to its pervasiveness and ability to passively log users’ context, smartphones offer the opportunity of collecting data at an unprecedented scale[Sensing and modeling Human behavoiur] many businesses have formed off of using that data. Targeted advertisements mainly started to grow popular on the television where what ad played was related to the channel, time and maybe location but now systems embedded inside Social media allows them to store and sell your personal data to many businesses especially advertiser that use your data to only advertise things they think the user will buy based off of the users personal data. Through the use of millions of users data its possible to showcase statistics that help predict the popular trends and what is gaining popularity, this is an important staticstic to businesses that sell 'trend items' which are products that are only worth selling during the time period they are relevant, examples are fidget spinners or products that have seen a increase in popularity due to an event. Tracking and storing user data is a specialised system for social media that funds the free experience on most social media that allows many of us to interact and share media with eachother without any monitary cost.
\newline
\section{Conclusion}
\section{Glossary}
@article{DBLP:journals/corr/abs-1801-08997,
author    = {Scott A. Carter and
               Pernilla Qvarfordt and
               Matthew Cooper and
               Aki Komori and
               Ville M{\"{a}}kel{\"{a}}},
title     = {Tools for online tutorials: comparing capture devices, tutorial representations,
               and access devices},
journal   = {CoRR},
volume    = {abs/1801.08997},
year      = {2018}
}

@article{DBLP:journals/corr/ShaoCVFM17,
  author    = {Chengcheng Shao and
               Giovanni Luca Ciampaglia and
               Onur Varol and
               Alessandro Flammini and
               Filippo Menczer},
  title     = {The spread of fake news by social bots},
  journal   = {CoRR},
  volume    = {abs/1707.07592},
  year      = {2017}
}

@article{DBLP:journals/corr/MehrotraM17,
  author    = {Abhinav Mehrotra and
               Mirco Musolesi},
  title     = {Sensing and Modeling Human Behavior Using Social Media and Mobile
               Data},
  journal   = {CoRR},
  volume    = {abs/1702.01181},
  year      = {2017}
}

@article{DBLP:journals/corr/abs-1901-02701,
  author    = {Agnese Chiatti and
               Dolzodmaa Davaasuren and
               Nilam Ram and
               Prasenjit Mitra and
               Byron Reeves and
               Thomas Robinson},
  title     = {Guess What's on my Screen? Clustering Smartphone Screenshots with
               Active Learning},
  journal   = {CoRR},
  volume    = {abs/1901.02701},
  year      = {2019}
}

ITU World Telecommunication/ICT Indicators Database (2013).

\end{document}
 