\documentclass[12pt, times new roman]{article}
\title{\textbf{COMP1010 Assignment 1}}
\author{Ka Yin Zhang}
\begin{document}
\maketitle
\newpage
\section{How does the development of specific hardware (VR and AR) aid with digital media?}
The development of digital media has become a vast resource that enables users to communicate with devices such as smartphones in various methods. It appears in many forms; whether it is browsing social media, reading the latest news or experimenting a new way to learn, each with its own different functions and intelligence that assists in the way users interact with each other, sharing ideas and connecting to the world. Already, it has become a part of society’s everyday life, and as it constantly changes, it is essential to continue to adapt with new, emerging hardware and technology so not only can society expand their media range but develop society as a whole. Thus, incorporating digital media using different forms of hardware within virtual reality or augmented reality can aid in effective and reliable communication. 
\newline Virtual reality (also known as VR) allows the user to experience a three-dimensional environment through a computer-generated setup. Hardware can include headsets and gloves to simulate most of a users’ senses such as sight, hearing and touch. It provides an immersive experience; depending on the level of immersion, the efficiency of the hardware differs. Combining digital media with virtual reality provides many new opportunities for a variety of users for many purposes including journalism. 
\newline For example, virtual reality creates a different avenue in how readers and viewers interact with journalism. In 2016, The Guardian released their virtual reality project: “6 x 9: A Virtual Experience of Solitary Confinement”, where viewers are able to experience life in solitary confinement. To participate in this interactive journey, the hardware can be easily accessed with a smartphone by downloading the app, and to assist in the viewer’s immersion, headphones and a mobile virtual reality headset can bring out their sight and hearing in the experience. While hardware seems limited and not as complex, this is necessary to meet the target audience of the everyday individual and mainly focus on the fundamental senses for this subject. Now, the viewer becomes the victim of solitary confinement so they can see what the victims saw through the enhanced vision of the phone through the headset; the 360º angle feature enables them to see a three-dimensional point of view of the uncomfortable space they see whenever they turn their head and the headphones creates the solemn mood followed by the storytelling. Rather than simply reading their story, this alternative option exposes the viewer to this unique platform and can empathise with the victims. Hence, this project displays how ideas can be communicated through this engaging delivery of journalism, allowing viewers to immerse into multiple perspectives.
\newline Augmented reality (AR) is where data and material in physical reality can be incorporated into digital forms. It often involves a device that detects and collects information from the physical reality where it can then analyse, process and project the results onto the device. This can provide more thorough ideas and improve aspects in the way society communicates ideas. Therefore, integrating augmented reality into digital media can strengthen its ability to communicate by converting information into a device which has been utilised in education.
\newline The use of augmented reality in education not only implements a new learning platform but can also encourage motivation in students to enjoy learning. In Indonesia, an augmented reality activity was developed to tackle the lack of motivation in history classes. Students are to use a smartphone where the camera can scan a card of a historic building and it would then generate a 3D model of that building with written information about it. This allows students to thoroughly examine the building and analyse the information, making learning more interactive. Using smartphones as the augmented reality device supports this simple, interactive learning activity because of its size and reliability as well as an alternative method of using pen and paper. 
\newline Digital media play an important part in effective and reliable communication. Utilising virtual reality and augmented reality systems improves the way with how things are normally done, inviting users to take in a new experience; all with a device where most people have access to: a smartphone. As mentioned before, a smartphone allows users to experience virtual reality and augmented reality that immerses them in a story and discover a new way to learn. Thus, combining digital media with virtual reality and augmented reality systems enhances the users’ ability to communicate ideas, interact with real life situations through simulation and creates a new platform to play and learn with emerging technology.


\end{document}